%% josis.tex 1.4   2016-09-15    JoSIS latex template
%------------------------------------------------------------------
% Filename: josis_template.tex
%
% This file is intended as a template for typesetting articles for the
%
%                        Journal of Spatial Information Science.
%
% Please edit this template to generate your own formatted manuscripts
% for submission to JOSIS. See http://josis.org for further details.
%


%%% JOSIS checks in typesetting
%%% * All titles and sections lower case *EXCEPT short title  [ ]
%%% * Remove author postal addresses, only have geographic places and institutions [ ] 
%%% * Consistent use of Section, Figure, Table (capitalized and in full) [ ]
%%% * 10 keywords (and all lower case) [ ]
%%% * Remove all avoidable footnotes [ ]
%%% * Use double quotation marks (``'' not "" or `') [ ]
%%% * Punctuation inside quotations [ ]
%%% * E.g. and i.e. followed by comma [ ]
%%% * cf. followed by tilde [ ]
%%% * Itemize and enumerate correctly punctuated [e.g., "1. x, 2. y, and 3. x." ]
%%% * And/or lists using American English punctuation (e.g., "x, y, and z") [ ] 
%%% * Bibliography (e.g., en-dashes for number ranges, consistent "Proc.~" for Proceedings of..., etc.) []
%%% * Acknowledgment style use section* [ ] 
%%% * et al. no italics, but with dot  [ ] 
%%% * All captions end with full stop  [ ] 
%%% * Table captions under, not over table  [ ]
%%% * Adjust urls with burlalt [ ] 
%%% * Check correct use of hyphens, emdashes, endashes  [ ]
%%% * Perform spell check  [ ] 

%%% JOSIS checks directly before publication 
%%% Check DOI, page numbers on article and web site. [ ]
%%% Update web site with final title, abstract, keywords. [ ] 
%%% Build with distiller for DOI links. [ ]


% Required documentclass definition for JOSIS
\documentclass{josis}
\usepackage{hyperref}
\usepackage[hyphenbreaks]{breakurl}
\usepackage{booktabs}
\usepackage{stmaryrd}
\usepackage[T1]{fontenc}
\usepackage[numbers,sort&compress]{natbib}

% Suggested packages for algorithm formatting
\usepackage{algorithm}
%\usepackage{algorithmic}
\usepackage{algpseudocode}


\usepackage[table]{xcolor}
\usepackage{amssymb,amsmath}
\usepackage{longtable}
\usepackage{listings}
\usepackage{calc}

% Added for Pandoc-crossref friendly \cref commands
\usepackage{cleveref}

% Pandoc compatibility macros
\providecommand{\tightlist}{}
\newcommand{\real}[1]{#1}

% Text wrapping helpers
\usepackage{fvextra}
\DefineVerbatimEnvironment{Verbatim}{Verbatim}{breaklines=true, breakanywhere=true}

% Force aggressive line breaking for all code listings
\lstset{
  breaklines=true,
  breakatwhitespace=false,
  breakanywhere=true,
  columns=fullflexible,
  basicstyle=\small\ttfamily,
  keepspaces=true
}

\renewcommand{\topfraction}{0.9} 
\renewcommand{\textfraction}{0.1}

% Page setup and overhangs
\sloppy
\widowpenalty=10000
\clubpenalty=10000
\hyphenpenalty=75

% Article details for accepted manuscripts will be added by editorial staff
% Omit year if article in press
% Omit number if article under review
\josisdetails{%
   number=N, year=YYYY, firstpage=xx, lastpage=yy, 
  doi={10.5311/JOSIS.YYYY.II.NNN},
   received={December 24, 2015}, 
   returned={February 25, 2016},
   revised={July 13, 2016},
   accepted={September 5, 2016}, }

\newcommand{\mydoi}[1]{\href{http://dx.doi.org/#1}{doi:\protect\detokenize{#1}}}

\urlstyle{rm}
\makeatletter
% Inspired by http://anti.teamidiot.de/nei/2009/09/latex_url_slash_spacingkerning/
% but slightly less kern and shorter underscore
\let\UrlSpecialsOld\UrlSpecials
\def\UrlSpecials{\UrlSpecialsOld\do\/{\Url@slash}\do\_{\Url@underscore}}%
\def\Url@slash{\@ifnextchar/{\kern-.11em\mathchar47\kern-.2em}%
    {\kern-.0em\mathchar47\kern-.08em\penalty\UrlBigBreakPenalty}}
\def\Url@underscore{\nfss@text{\leavevmode \kern.06em\vbox{\hrule\@width.3em}}}
\makeatother

\hypersetup{
colorlinks=true,
linkcolor=black,
citecolor=black,
urlcolor=black
} 

% Add the running author and running title information
\runningauthor{\begin{minipage}{.9\textwidth}\centering Author1, Author2\end{minipage}}
\runningtitle{Short Title for JOSIS Article}

% Document begins
\begin{document}

% Title, author, abstract, keywords and metadata
\title{Benchmarking Large Language Models for Geolocating Colonial Virginia Land Grants}
\author{Ryan Mioduski}\affil{Independent Researcher}

\maketitle

\keywords{historical GIS, large language models, geoparsing, colonial Virginia, land grants, digital humanities, spatial history, geolocation}

\begin{abstract}
Virginia's seventeenth- and eighteenth-century land patents survive almost exclusively as narrative metes-and-bounds descriptions in printed abstract volumes such as \emph{Cavaliers and Pioneers} (C\&P). I present the first systematic study of whether state-of-the-art large language models (LLMs) can convert these prose abstracts into usable latitude/longitude coordinates at research grade. I digitized, transcribed, and openly released a corpus of 5,471 Virginia patent abstracts (1695–1732), accompanied by a rigorously annotated ground-truth dataset of 45 authoritatively georeferenced test cases. I benchmark six OpenAI models spanning three architecture families under two prompting paradigms: (i) one-shot "direct-to-coordinate" and (ii) tool-augmented chain-of-thought that invokes external geocoding APIs.

On the verified grants, the best purely textual model (OpenAI o3) achieves a mean great-circle error of 23.4 km (median 14.3 km), a 67\% improvement over a professional GIS baseline (71.4 km), while cutting cost and latency by roughly two and three orders of magnitude, respectively. The ultracheap GPT-4o variant locates patents with 28 km mean error at USD 1.09 per 1,000, only slightly less accurate yet ~100× cheaper. Contrary to expectations, granting LLMs external geocoding tools neither improves accuracy nor consistency.

These results show that off-the-shelf LLMs can georeference early-modern land records faster, cheaper, and as accurately as traditional GIS workflows, opening a scalable pathway to spatially enable colonial archives—and, in turn, to reassess settlement dynamics, plantation economies, and Indigenous dispossession with quantitative precision.
\end{abstract}

% Main body generated via Pandoc
\input{content.tex}

% References
\bibliographystyle{josisacm}
\bibliography{refs}

\end{document}